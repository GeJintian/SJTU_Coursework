\documentclass[12pt,a4paper]{article}
%\usepackage{ctex}
\usepackage{amsmath,amscd,amsbsy,amssymb,latexsym,url,bm,amsthm}
\usepackage{epsfig,graphicx,subfigure}
\usepackage{enumitem,balance}
\usepackage{wrapfig}
\usepackage{mathrsfs,euscript}
\usepackage[x11names,svgnames,dvipsnames]{xcolor}
\usepackage{hyperref}
\usepackage[vlined,ruled,commentsnumbered,linesnumbered]{algorithm2e}
\usepackage{listings}
\usepackage{multicol}
%\usepackage{fontspec}

\renewcommand{\listalgorithmcfname}{List of Algorithms}
\renewcommand{\algorithmcfname}{Alg.}
\newtheorem{theorem}{Theorem}
\newtheorem{lemma}[theorem]{Lemma}
\newtheorem{proposition}[theorem]{Proposition}
\newtheorem{corollary}[theorem]{Corollary}
\newtheorem{exercise}{Exercise}
\newtheorem*{solution}{Solution}
\newtheorem{definition}{Definition}
\theoremstyle{definition}


%\numberwithin{equation}{section}
%\numberwithin{figure}{section}

\renewcommand{\thefootnote}{\fnsymbol{footnote}}

\newcommand{\postscript}[2]
 {\setlength{\epsfxsize}{#2\hsize}
  \centerline{\epsfbox{#1}}}

\renewcommand{\baselinestretch}{1.0}

\setlength{\oddsidemargin}{-0.365in}
\setlength{\evensidemargin}{-0.365in}
\setlength{\topmargin}{-0.3in}
\setlength{\headheight}{0in}
\setlength{\headsep}{0in}
\setlength{\textheight}{10.1in}
\setlength{\textwidth}{7in}
\makeatletter \renewenvironment{proof}[1][Proof] {\par\pushQED{\qed}\normalfont\topsep6\p@\@plus6\p@\relax\trivlist\item[\hskip\labelsep\bfseries#1\@addpunct{.}]\ignorespaces}{\popQED\endtrivlist\@endpefalse} \makeatother
\makeatletter
\renewenvironment{solution}[1][Solution] {\par\pushQED{\qed}\normalfont\topsep6\p@\@plus6\p@\relax\trivlist\item[\hskip\labelsep\bfseries#1\@addpunct{.}]\ignorespaces}{\popQED\endtrivlist\@endpefalse} \makeatother


\definecolor{codegreen}{rgb}{0.44,0.68,0.28}
\definecolor{codegray}{rgb}{0.5,0.5,0.5}
\definecolor{codepurple}{rgb}{0.58,0,0.82}
\definecolor{backcolour}{rgb}{0.96,0.96,0.96}

\lstset{
language=C++,
frame=shadowbox,
keywordstyle = \color{blue}\bfseries,
commentstyle=\color{codegreen},
tabsize = 4,
backgroundcolor=\color{backcolour},
numbers=left,
numbersep=5pt,
breaklines=true,
emph = {int,float,double,char},emphstyle=\color{orange},
emph ={[2]const, typedef},emphstyle = {[2]\color{red}} }



\begin{document}
\noindent

%========================================================================
\noindent\framebox[\linewidth]{\shortstack[c]{
\Large{\textbf{Lab06-Heaps and BST}}\vspace{1mm}\\
VE281 - Data Structures and Algorithms, Xiaofeng Gao, TA: Li Ma, Autumn 2019}}
%CS26019 - Algorithm Design and Analysis, Xiaofeng Gao, Autumn 2019}}
\begin{center}
\footnotesize{\color{red}$*$ Please upload your assignment to website. Contact webmaster for any questions.}

\footnotesize{\color{blue}$*$ Name:\_\_\_\_\_\_\_\_\_  \quad Student ID:\_\_\_\_\_\_\_\_\_ \quad Email: \_\_\_\_\_\_\_\_\_\_\_\_}
\end{center}


\begin{enumerate}

\item \textbf{D-ary Heap.} D-ary heap is similar to binary heap,but (with one possible exception) each non-leaf node of d-ary heap has $d$ children, not just $2$ children.

\begin{enumerate}
\item How to represent a d-ary heap in an array?
\item What is the height of the d-ary heap with $n$ elements? Please use $n$ and $d$ to show.
\item Please give the implementation of insertion on the min heap of d-ary heap, and show the time complexity with $n$ and $d$.
\end{enumerate}

\begin{lstlisting}[language=C++]
// Input: an integer k
// Output: null
void enqueue(int k)
{
	// TODO;
}
\end{lstlisting}

%\begin{solution}
%	Uncomment this block to write your solution.
%\end{solution}

\item \textbf{Median Maintenance.} Input a sequence of numbers $x_1,x_2...,x_n$, one-by-one. At each time step $i$, output the median of $x_1,x_2...,x_i$. How to do this with $O(\log i)$ time at each step $i$? Show the implementation.

\item  \textbf{BST}. Two elements of a binary search tree are swapped by mistake. Recover the tree without changing its structure. Implement with a constant space.

\begin{lstlisting}[language=C++]
/**
 * Definition for binary tree
 * struct TreeNode {
 *     int val;
 *     TreeNode *left;
 *     TreeNode *right;
 *     TreeNode(int x) : val(x), left(NULL), right(NULL) {}
 * };
 */
void recoverTree(TreeNode *root)
{
	// TODO;
}
\end{lstlisting}

%\begin{solution}
%	Uncomment this block to write your solution.
%\end{solution}

\item  \textbf{BST}. Input an integer array, then determine whether the array is the result of the post-order traversal of a binary search tree. If yes, return Yes; otherwise, return No. Suppose that any two numbers of the input array are different from each other. Show the implementation.

\begin{lstlisting}[language=C++]
// Input: an integer array
// Output: yes or no
bool verifySquenceOfBST(vector<int> sequence)
{
	// TODO;
}
\end{lstlisting}

%\begin{solution}
%	Uncomment this block to write your solution.
%\end{solution}

\end{enumerate}

%========================================================================
\end{document}
